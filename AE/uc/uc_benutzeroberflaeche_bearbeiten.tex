\textbf{Benutzeroberflaeche bearbeiten}
  \begin{itemize}
  \item \textit{Ziel:} Der Nutzer ruft eine Funktionalitaet der Benutzeroberflaeche auf.
  \item \textit{Einordnung:} Hauptfunktion
  \item \textit{Vorbedingung:} Die Applikation wurde gestartet und das System zeigt die Benutzeroberflaeche an.
  \item \textit{Nachbedingung:} Die Benutzeroberfl\"ache wurde ver\"andert.
  \item \textit{Nachbedingung im Fehlerfall:} 
  \item \textit{Hauptakteure:} Nutzer
  \item \textit{Nebenakteure:} System
  \item \textit{Ausl\"oser:} Der Nutzer m\"ochte eine Funktionalitaet der Benutzeroberflaeche ausfuehren.
  \item \textit{Standardablauf:}
    \begin{enumerate}[label=(\arabic*)]
    \item Das System ruft den UC Case Beenden auf.
    \end{enumerate}
  \item \textit{Verzweigungen:}
    \begin{enumerate}[label=(1\alph*)]
\item Das System ruft den UC Case Definitionsbereich aendern auf.
\item Das System ruft den UC Case Reset auf
\item Das System ruft den UC Case Interpolationsmethode aendern auf
    \end{enumerate}
  \end{itemize}