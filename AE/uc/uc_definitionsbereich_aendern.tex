\textbf{Definitionsbereich \"andern}
  \begin{itemize}
  \item \textit{Ziel:} Der Definitionsbereich soll ge\"andert werden
  \item \textit{Einordnung:}  Hauptfunktion
  \item \textit{Vorbedingung:} Die Applikation ist ge\"offnet und zeigt die Zeichenfl\"ache sowie die Benutzeroberfl\"ache an.
  \item \textit{Nachbedingung:} Die Applikation zeigt die aktualisierte Zeichenfl\"ache sowie Benutzeroberfl\"ache mit den neuen Grenzen an.
  \item \textit{Nachbedingung im Fehlerfall:} Die Applikation zeigt eine Fehlermeldung an und ver\"andert die Zeichenfl\"ache nicht.
  \item \textit{Hauptakteure:} Nutzer
  \item \textit{Nebenakteure:}
  \item \textit{Ausl\"oser:} Der Nutzer will den Definitionsbereich \"andern
  \item \textit{Standardablauf:}
    \begin{enumerate}[label=(\arabic*)]
    \item Der Nutzer klickt mit der linken Maustaste auf eins der Textfelder auf der Benutzeroberfl\"ache unter der \"Uberschrift "'Definitionsbereich".
    \item Der Nutzer \"andert den Wert im Textfeld.
    \item Der Nutzer bet\"atigt mit der linken Maustaste den Button 'Definitionsbereich setzen'.
    \item Die Applikation pr\"uft, ob Zahlen eingegeben wurden.
    \item Die Applikation pr\"uft, ob $ x_{min} > x_{max}$ ist. 
    \item Die Applikation pr\"uft, ob $ y_{min} > y_{max}$ ist.
    \item Die Applikation \"andert den Definitionsbereich auf die gew\"unschten Werte.
    \end{enumerate}
      \item \textit{Verzweigungen:}
     \begin{enumerate}[label=(2a\arabic*)]            	\setcounter{enumii}{3}
      \item Der Nutzer will noch einen anderen Wert des Definitionsbereiches \"andern.
   	\item Der Nutzer geht zu Schritt '1' zur\"uck.
                \end{enumerate}      
    \begin{enumerate}[label=(4a\arabic*)]
      \item Der Nutzer hat eine ung\"ultige Eingabe get\"atigt.
      \item Die Applikation gibt eine Fehlermeldung aus.
      \end{enumerate}
                      \begin{enumerate}[label=(5a\arabic*)]
    \setcounter{enumii}{4}
    \item Die Applikation erkennt, dass  '$ x_{min} > x_{max}$'  ist.
      \item Die Applikation gibt eine Fehlermeldung aus.
      \end{enumerate}
           \begin{enumerate}[label=(6a\arabic*)]
     \setcounter{enumii}{5}
    \item Die Applikation erkennt, dass  '$ y_{min} > y_{max}$'  ist.
    \item Die Applikation gibt eine Fehlermeldung aus.
      \end{enumerate}
  \end{itemize}
