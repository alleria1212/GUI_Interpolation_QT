\textbf{Reset}
  \begin{itemize}
  \item \textit{Ziel:} Es sollen alle St\"utzstellen und die Kurve von der Zeichenfl\"ache gel\"oscht, sowie der Definitionsbereich und die Interpolationsart auf den Defaultmodus (Lineare Spline Interpolation ([0,100]x[0,50]) zur\"uckgesetzt werden.
  \item \textit{Einordnung:} Hauptfunktion
  \item \textit{Vorbedingung:} Die Applikation ist ge\"offnet und zeigt die Zeichenfl\"ache sowie die Benutzeroberfl\"ache an.
  \item \textit{Nachbedingung:} Die Applikation hat alle St\"utzstellen auf der Zeichenfl\"ache gel\"oscht und der Definitionsbereich sowie die Interpolationsart sind im Defaultmodus.
  \item \textit{Nachbedingung im Fehlerfall:} 
  \item \textit{Hauptakteure:} Nutzer
  \item \textit{Nebenakteure:}
  \item \textit{Ausl\"oser:} Der Nutzer m\"ochte die Applikation auf den Defaultmodus zur\"ucksetzen.
  \item \textit{Standardablauf:}
    \begin{enumerate}[label=(\arabic*)]
    \item Der Nutzer klickt mit der linken Maustaste auf den Button 'Reset', welcher sich auf der Benutzeroberfl\"ache befindet.
    \item Die Applikation stellt den Defaultmodus wieder her und l\"oscht die St\"utzstellen.
   % \item Die Applikation entfernt alle St\"utzstellen.							%Sollen die Aktivit\"aten des Systemes angezeigt werden?
    %\item Die Applikation ruft den UC Case Interpolieren auf
   %\item Die Applikation setzt den Definitionsbereich auf [0,50]x[0,100].
  %  \item Die Applikation setzt die Interpolationsart auf  'Lineare Splines'.
    
    \end{enumerate}
 \end{itemize}