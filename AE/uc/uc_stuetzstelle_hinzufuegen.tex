\textbf{St\"utzstelle hinzuf\"ugen}
  \begin{itemize}
  	\item \textit{Ziel:} Die Interpolation soll um eine St\"utzstelle erweitert werden.
  	\item \textit{Einordnung:} Hauptfunktion
  	\item \textit{Vorbedingung:} Die Applikation ist ge\"offnet und zeigt die Zeichenfl\"ache sowie die Benutzeroberfl\"ache an.
  	\item \textit{Nachbedingung:} Die Applikation zeigt die neu berechnete Interpolation mit der hinzugef\"ugten St\"utzstelle auf der Zeichenfl\"ache an.
  \item \textit{Nachbedingung im Fehlerfall:} 
  \item \textit{Hauptakteure:} Nutzer
  \item \textit{Nebenakteure:} 
  \item \textit{Ausl\"oser:} Der Nutzer m\"ochte eine St\"utzstelle auf der Zeichenfl\"ache hinzuf\"ugen.
  \item \textit{Standardablauf:}
    \begin{enumerate}[label=(\arabic*)]
    \item Der Nutzer klickt mit der linken Maustaste auf eine beliebige Stelle der Zeichenfl\"ache.
    \item Die Applikation erzeugt eine St\"utzstelle auf dem angeklickten Punkt.
    \item Die Applikation f\"uhrt die Interpolation mit den neuen St\"utzstellen aus.
    \item Die Applikation zeigt die neu berechnete Interpolation auf der Zeichenfl\"ache an.
    \end{enumerate}
  \end{itemize}
