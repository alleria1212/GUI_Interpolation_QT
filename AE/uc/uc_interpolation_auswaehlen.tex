\textbf{Art der Interpolation \"andern}
  \begin{itemize}
  \item \textit{Ziel:} Der Nutzer will die Interpolationsart \"andern.
  \item \textit{Einordnung:} Hauptfunktion
  \item \textit{Vorbedingung:} Die Applikation ist ge\"offnet und zeigt die Zeichenfl\"ache sowie die Benutzeroberfl\"ache an.
  \item \textit{Nachbedingung:} Die Applikation zeigt die neu berechnete Kurve auf der Zeichenfl\"ache an.
  \item \textit{Nachbedingung im Fehlerfall:} 
  \item \textit{Hauptakteure:} Nutzer
  \item \textit{Nebenakteure:} 
  \item \textit{Ausl\"oser:} Der Nutzer will die Interpolationsart \"andern.
  \item \textit{Standardablauf:}
    \begin{enumerate}[label=(\arabic*)]
    \item Der Nutzer klickt mit der linken Maustaste auf die Auswahlbox mit der \"Uberschrift "'Interpolationsart", welche sich auf der Benutzeroberfl\"ache befindet.
    \item Die Applikation zeigt alle m\"oglichen Interpolationsmethoden in der Auswahlbox an.
    \item Der Nutzer w\"ahlt mit der linken Maustaste eine Interpolationsart aus.
    \item Die Applikation f\"uhrt die Interpolation mit der ausgew\"ahlten Interpolationsmethode aus.
    \item Die Applikation zeigt die neu berechnete Interpolation auf der Zeichenfl\"ache an.
    \end{enumerate}
  \end{itemize}