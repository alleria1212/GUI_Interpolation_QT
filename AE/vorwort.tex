\chapter{Vorwort}
\label{ch:1}

\section{Aufgabenstellung und Struktur des Dokuments}
\label{sec:1.1}

Sehr geehrte Damen und Herren, \\
\hfill \\
\noindent Bitte entwerfen und implementieren sie eine Applikation, die die Interpolation von Werten
zwischen festgelegten Punkten in einer X-Y-Ebene erlaubt. Die Interpolation soll genutzt
werden, um die Punkte graphisch mit einer Linie zu verbinden.\\
Die Interpolationsroutine soll f\"ur m Kontrollpunkte die X- und Y-Koordinaten von n St\"utzstellen berechnen, die \"aquidistant (Abstand dx) im Intervall [xmin; xmax] verteilt liegen. \\
\noindent Ausschlaggebend f\"ur eine gute Integration der Software in unseren Arbeitsprozess ist, dass
mithilfe einer graphischen Benutzerschnittstelle die Randbedingungen und die Interpolationsart eingestellt werden k\"onnen.\\
\noindent Weiterhin soll eine Visualisierung der resultierenden Interpolation auf der Benutzeroberfl\"ache m\"oglich sein. \\ \\
\noindent Wir freuen uns auf eine Zusammenarbeit. \\ 
\clearpage

\section{Projektmanagement}
\label{sec:1.2}

Tom Witter:
\begin{itemize}
\item Benutzeranforderungen
\item Implementation InterpolationControl
\end{itemize}
Stefan Jeske:
\begin{itemize}
\item Benutzerdokumentation
\item Implementation MainWindow, User Interface
\end{itemize}
Daniel Partida:
\begin{itemize}
\item Aufgabenstellung und Struktur des Dokuments
\item Implementation MainWindow
\end{itemize}
Chun-Kan Chow:
\begin{itemize}
\item Entwicklerdokumentation
\item Implementation Interpolationsarten (CubicSpline, PolynomialInterpolation)
\end{itemize}
Alle:
\begin{itemize}
\item Aktivit\"atsdiagramme
\item UC Beschreibungen
\item funktionale und nicht funktionale Anforderungen
\item Begriffsanalyse
\item Sequenzdiagramme
\item Klassendiagramm
\item Check
\end{itemize}


\section{Lob und Kritik}
\label{sec:1.3}

Danke Naumann f\"ur die Organisation des Projektes und die gewonnene Praxiserfahrung. ;-)

