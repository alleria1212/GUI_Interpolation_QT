\chapter{Entwicklerdokumentation}
\label{ch:5}

\section{Codestruktur}
Der Code besteht aus mehreren Klassen. Die Quelldateien befinden sich alle im gleichen Verzeichnis. Siehe Kapitel 4 f\"ur Installation. Im Folgenden wird die Codestruktur im Groben beschrieben. F\"ur eine detaillierte Dokumentation siehe Abschnitt 2 in diesem Kapitel.

\subsection{Klasse MainWindow}
Die Klasse {\tt MainWindow} ist f\"ur die graphische Darstellung der Benutzeroberfl\"ache zust\"andig. Der Quellcode findet sich in \refsec{ssec:6.1.1} und in den Dateien {\tt mainwindow.h} und {\tt mainwindow.cpp}.

\subsection{Klasse InterpolationControl}
Die Klasse {\tt InterpolationControl} verwaltet die zur verf\"ugung stehenden Interpolationsarten. Der Quellcode befindet sich in \refsec{ssec:6.1.2} und in den Dateien {\tt interpolationcontrol.h} und {\tt interpolationcontrol.cpp}

\subsection{Klasse Interpolation}
Diese Klasse stellt eine abstrakte Klasse zur Verf\"ugung, welche abgeleitet wird um die einzelnen Interpolationsmethoden zu implementieren. Die Standardm\"a\ss ig verf\"ugbaren Methoden sind lineare Splines, kubische Splines  und Polynominterpolation (Grad m-1, wobei m die Anzahl der St\"utzstellen ist.) Der Quellcode befindet sich jeweils im Anhang \refsec{ssec:6.1.3} und in den Dateien {\tt interpolation.h} und {\tt interpolation.h}.\\ 
Der Quellcode der implementierten Verfahren befindet sich in \refsec{ssec:6.1.4} bis \refsec{ssec:6.1.6}


\section{Detaillierte Dokumentation des Codes}
In dem Unterverzeichnis {\tt ./Interpolation/doc/} befindet sich die Konfigurationsdatei {\tt Doxyfile} f\"ur die Erstellung der doxygen Dokumentation in {\tt HTML} oder \LaTeX\ Format. Zum Erstellen der Dokumentation muss der Befehl {\tt doxygen Doxyfile} in dem Verzeichnis {\tt ./Interpolation/doc/} aufgerufen werden. Es werden zwei neue Verzeichnisse {\tt ./Interpolation/doc/html/} und {\tt ./Interpolation/doc/latex/} erstellt. Um die {\tt HTML} Version anzuzeigen muss in den Ordner {\tt html} gewechselt werden und die Datei {\tt index.html} mit dem Standardbrowser ge\"offnet werden. Um die \LaTeX\ Dokumentation anzeigen zu lassen muss der Befehl {\tt make} in dem Ordner {\tt latex} ausgef\"uhrt werden. Anschlie\ss end kann die PDF Datei mit einem entsprechenden Programm gelesen werden.

\section{Hinzuf\"ugen einer neuen Interpolationsart}
Das Hinzuf\"ugen einer neuen Interpolationsart l\"asst sich in die folgenden Schritte unterteilen. 

\subsection{Ableiten der Klasse Interpolation}
Zun\"achst muss, um eine neue Interpolationsart zu implementieren, die Klasse Interpolation in der folgenden Form abgeleitet werden.

\begin{lstlisting}
#include <interpolation.h>

class myInterpolation : public Interpolation
{
public:
	void StartInterpolation(
            const QVector<double>& xPos,
            const QVector<double>& yPos,
            QVector<double>& xCurve,
            QVector<double>& yCurve
            );
};
\end{lstlisting}

Die eigentliche Interpolation wird durch die Schnittstelle {\tt StartInterpolation(...)} definiert, welche in den Variablen {\tt xCurve, yCurve} die zu {\tt xPos,yPos} geh\"orenden interpolierten Punkte speichert, die zum Zeichnen der Kurve verwendet werden.\\
Diese Methode muss von dem Benutzer in folgender Form implementiert werden:

\begin{lstlisting}

void myInterpolation::StartInterpolation(
            const QVector<double>& xPos,
            const QVector<double>& yPos,
            QVector<double>& xCurve,
            QVector<double>& yCurve
            )
{
	...
	<myImplementation>
	...
}

\end{lstlisting}

\subsection{Hinzuf\"ugen der Interpolationsart in den Quellcode: InterpolationControl}
Angenommen, die neue Interpolationsart wurde in den Dateien {\tt myinterpolation.h} und {\tt myinterpolation.cpp} gespeichert, muss zun\"achst die folgende Zeile zu den {\tt includes} der {\tt interpolationcontrol.h} und ein neues Objekt in den {\tt private:} Teil der Klasse hinzugef\"ugt werden. 

\begin{lstlisting}
...
#include "myinterpolation.h"
...
\end{lstlisting}

\begin{lstlisting}
Interpolation * myInterpolationModel;
\end{lstlisting}

Danach muss in der Datei {\tt interpolationcontrol.cpp} in dem Konstruktor ein Objekt vom Typ {\tt myInterpolation} erzeugt  und in der Funktion {\tt changeActiveModel(...)} ein weiterer switch-case hinzugef\"ugt werden:

\begin{lstlisting} 
InterpolationControl::InterpolationControl()
{
	...
	myInterpolationModel = new myInterpolation();
	...
}

...

void InterpolationControl::changeActiveModel(int _activeModel)
{
	...
	switch(_activeModel)
	{
		...
		case 3 : activeModel = myInterpolationModel;
			str = "myInterpolation ausgewaehlt";
			break;
	}
}
\end{lstlisting}

\subsection{Hinzuf\"ugen der Interpolationsart in den Quellcode: MainWindow}
Zuletzt muss noch eine Zeile am Ende des Konstruktors der Klasse {\tt MainWindow} eingef\"ugt werden.
\begin{lstlisting}
MainWindow::MainWindow(QWidget *parent) :
    QMainWindow(parent),
    ui(new Ui::MainWindow)
{
	...
	ui->interpolation_select->addItem("myInterpolation");
}
\end{lstlisting}
Nun befindet sich ein weiterer Eintrag {\tt myInterpolation} in der Auswahlbox auf der Benutzeroberfl\"ache und kann ausgew\"ahlt werden, um die Punkte auf der Zeichenfl\"ache mit {\tt myInterpolation} zu verbinden.

\section{Software-Tests}
Alle Anwendungsf\"alle wurden manuell auf Funktionalit\"at getestet. Die folgenden Anwendungsf\"alle wurden erfolgreich ohne Fehlermeldungen getestet:
\begin{itemize}
  \item Interpolationsart \"andern
  \item Reset
  \item Beenden
  \item St\"utzstelle l\"oschen
  \item St\"utzstelle hinzuf\"ugen
\end{itemize}
Der Anwendungsfall 'Definitionsbereich \"andern' wurde erfolgreich mit allen Fehlersituationen getestet, die in der Benutzerdokumentation ausf\"uhrlich beschrieben sind.








